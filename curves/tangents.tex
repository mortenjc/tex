\section{Tangents}

We have seen that a function has associated with it a $y$-value for every $x$-value. But curves have other 
properties. For example the slope varies with $x$ and this slope is called the \myindx{tangent} 
of the curve, and we can calculate the slope when we know the function.
For functions of the type $y=f(x)$, the slope is defined naturally as the 
small changes in the $y$-direction caused by a small change in $x$. This is 
called the \myindx{derivative} of the function. There are several ways of denoting the 
derivative of a function $y=f(x)$: 

$$
    slope = \dd{f(x)}{x} = f'(x) = f' = y' = \dd{y}{x}
$$ 
and if the variable is measuring time, $t$ as in $y=g(t)$ the derivatime is
sometimes denoted $\dot{g}(t)$ or $\dot{g}$.

When the function is parametrized $(x,y)=(f(t),g(t))$, we differentiate each
function and obtain a \myindx{tangent vector}.

$$
     \vec{v}_t = \begin{pmatrix} f'(t) \\ g'(t) \end{pmatrix}
$$

\begin{figure}[h]
\begin{center}
\begin{tikzpicture}[scale=2]
    \txfig{0}{-0.5}{3.2}{1.4}
    \draw[very thick,color=blue, domain=0:3.1416,smooth] plot[prefix=pgffigs/mjc,id=tangent] function{sin(3*x)/(x**2 +1)};
    \pgfsetlinewidth{0.75pt}
    \draw[red, ->] (0,0) -- +(0.158,0.474);
    \fill (0,0) circle (1pt);

    \draw[red, ->] (0.5,0.798) -- +(0.453,-0.212);
    \fill (0.5,0.798) circle (1pt);

    \draw[red, ->] (0.785,0.437) -- +(0.249,-0.433);
    \fill (0.785,0.437) circle (1pt);

    \draw[red, ->] (1.571,-0.288) -- +(0.484,0.126);
    \fill (1.571,-0.288) circle (1pt);

    \draw[red, ->] (2.356,0.108) -- +(0.486,0.120);
    \fill (2.356,0.108) circle (1pt);

    \node[right] at (2.0,1.2) {\Large $y=\frac{\sin(3x)}{x^2 +1}$};
\end{tikzpicture}
\caption{\small A curve and its tangent vectors at selected points. }
\end{center}
\end{figure}
