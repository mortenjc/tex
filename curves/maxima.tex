\newcommand{\lcs}{\Gamma}
\newcommand{\mcs}[2]{\Gamma_{#1}^{#2}}

\chapter{Tensor Examples}
\label{sec:christoffel}

In this section we will give some examples of the Christoffel symbols (of 
both kinds) derived from specific coordinate systems. It will be demonstrated
that depending on the metric tensor components there can be few or many of either 
kind. 

When working with tensors and their \myindx{derivatives}, such as the ones 
needed for calculating the \myindx{Christoffel symbols}, it is very time 
consuming to do these calculations by hand. 
To demonstrate the complexity of this \myindx{algebraic manipulation} let us 
calculate $\Gamma^2_{12}$ for the \myindx{spherical coordinates} described in section \ref{sec:sphcoord}.

$$
    \Gamma^2_{12} = \frac{1}{2} g^{2m} \left( \frac{\partial g_{m1}}{\partial u^2}  
                 +\frac{\partial g_{m2}}{\partial u^1}
                 -\frac{\partial g_{12}}{\partial u^m}
       \right)
$$

since $u^1=r, u^2=\theta$, and $u^3=\phi$ we get, when summing over $m=1,2,3$

$$
  \begin{array}{llllll}
    \Gamma^2_{12}  &= &\frac{1}{2} g^{21} \left( \frac{\partial g_{11}}{\partial \theta}  
                 +\frac{\partial g_{12}}{\partial r}
                 -\frac{\partial g_{12}}{\partial r} \right) +
                 \frac{1}{2} g^{22} \left( \frac{\partial g_{21}}{\partial \theta}  
                 +\frac{\partial g_{22}}{\partial r}
                 -\frac{\partial g_{12}}{\partial \theta} \right) \\ 
             &&+ 
                 \frac{1}{2} g^{23} \left( \frac{\partial g_{31}}{\partial \theta}  
                 +\frac{\partial g_{32}}{\partial r}
                 -\frac{\partial g_{12}}{\partial \phi} \right) \\
          & = &\frac{1}{2} g^{22}\frac{\partial g_{22}}{\partial r} = \frac{1}{2}\frac{1}{r^2}(2r) = \frac{1}{r}
  \end{array}
$$
It turns out that all but one derivative of the metric tensor are zero, but this is not
always the case! Note that there are $N^3$ Christoffel symbols, so in three dimensions we need to 
do this 27 times performing $9*27=243$ derivatives!

Clearly the possibility for errors is significant, so why not let a symbolic math program handle the 
tiresome and error prone calculations? After all this has been done by professionals 
since 1966 \cite{misner}.  The following examples were created using 
\myindx{Maxima} \cite{maxima}. For source code examples see 
section \ref{sec:maxima}.

The Christoffel symbol of the first kind is defined as 

$$
    \Gamma_{ijk} = \frac{1}{2} \left( \frac{\partial g_{ik}}{\partial u^j}  
                 +\frac{\partial g_{jk}}{\partial u^i}
                 -\frac{\partial g_{ij}}{\partial u^k}
       \right)
$$
and the Christoffel symbol of the second kind is 
$$
    \Gamma^k_{ij} = \frac{1}{2} g^{km} \left( \frac{\partial g_{mi}}{\partial u^j}  
                 +\frac{\partial g_{mj}}{\partial u^i}
                 -\frac{\partial g_{ij}}{\partial u^m}
       \right)
$$


In the following we will calculate the Christoffel symbols of the first and second kind for 
a number of coordinate systems using these formulae.


\section{Cylindrical Coordinates}

From section \ref{sec:cylcoord} we have

$$
   \vec{r}(\rho,\phi,z) = (\rho \cos(\phi), \rho \sin(\phi), z)
$$

and 



\begin{eqnarray*}
\lcs_{122}=&\rho, \qquad \lcs_{221}=&-\rho
\end{eqnarray*}

$$
g^{ij} = \begin{pmatrix}
   1 &      0          & 0\cr 
   0 &\frac{1}{\rho^2} & 0\cr 
   0 &      0          & 1\cr 
\end{pmatrix}
$$

\begin{eqnarray*}
\mcs{12}{2}=&\frac{1}{\rho}, \qquad \mcs{22}{1}=&-\rho
\end{eqnarray*}


\section{\myindx{Spherical Coordinates}}
\label{sec:maxima_sph}

Of the 27 potential Christoffel symbols of the first kind, only six are 
nonzero and only three needs to be calculated due to symmetry properties.

\begin{twocol}{
\begin{eqnarray*}
\lcs_{122}=&r \\
\lcs_{133}=&r\,\cos ^2\theta \\
\lcs_{221}=&-r \\
\end{eqnarray*}}{
\begin{eqnarray*}
\lcs_{233}=&-r^2\,\cos \theta\,\sin \theta \\
\lcs_{331}=&-r\,\cos ^2\theta \\
\lcs_{332}=&r^2\,\cos \theta\,\sin \theta
\end{eqnarray*}}
\end{twocol}

The inverse of $g_{ij}$ is

$$ 
    g^{ij}=\begin{pmatrix}1&0&0\cr 0&\frac{1}{r^2}&0\cr 0&0&\frac{1}{r^2\,\cos ^2
 \theta}\cr \end{pmatrix}
$$


From this we can calculate the Christoffel symbols of second kind. 

\begin{twocol}{
\begin{eqnarray*}
\mcs{12}{2}=&\frac{1}{r} \\
\mcs{13}{3}=&\frac{1}{r} \\
\mcs{22}{1}=&-r \\
\end{eqnarray*}}{
\begin{eqnarray*}
\mcs{23}{3}=&-\frac{\sin \theta}{\cos \theta} \\
\mcs{33}{1}=&-r\,\cos ^2\theta \\
\mcs{33}{2}=&\cos \theta\,\sin \theta \\
\end{eqnarray*}}
\end{twocol}{


\section{Ellipsoidal Coordinates}
If we take the innocent spherical coordinates and change them in the following way

$$
   \vec{r}(u,v,w) = (a u \sin v \cos w, bu\sin v \sin w, c u \cos v)
$$
we get the \myindx{ellipsoidal coordinates}
where $r$ has been replaced by $u$, $\theta$ by $v$ and $\phi$ by $w$ we get for the 
metric tensor $g_{ij}=$


\begin{wide}
$$
   \begin{pmatrix}

  \sin^2v(b^2\sin ^2w+a^2\cos ^2w)+c^2\cos ^2v   &
  u\cos v \sin v(b^2\sin ^2w+a^2\cos ^2 w-c^2)  &
  \left(b^2-a^2\right)u\sin ^2v\cos w \sin w\cr 

  u\cos v \sin v(b^2\sin ^2w+a^2 \cos ^2w-c^2)  &
  u^2\cos^2v(b^2\sin ^2w+a^2\cos ^2w)+c^2u^2\sin ^2v  &
  \left(b^2-a^2\right)u^2 \cos v\sin v\cos w\sin w\cr 

  \left(b^2-a^2\right)u\sin ^2v \cos w\sin w   &
  \left(b^2-a^2\right)u^2\cos v\sin v\cos w \sin w   &
   a^2u^2\sin ^2v\sin ^2w+b^2u^2\sin ^2v\cos ^2w\cr 
   \end{pmatrix}
$$
\end{wide}
... and as you can see the page is barely wide enough to hold the result even with a tiny font!  But from
Maxima we get the following 15 Christoffel symbols of the first kind

\begin{eqnarray*}
\Gamma_{121}=&\cos v\,\sin v\,\left(b^2\,\sin ^2w+a^2\,\cos ^2w
 -c^2\right)\\
\Gamma_{122}=&u\,\left(b^2\,\cos ^2v\,\sin ^2w+a^2\,\cos ^2v\,
 \cos ^2w+c^2\,\sin ^2v\right)\\
\Gamma_{123}=&\left(b-a\right)\,\left(b+a\right)\,u\,\cos v\,
 \sin v\,\cos w\,\sin w\\
\Gamma_{131}=&\left(b-a\right)\,\left(b+a\right)\,\sin ^2v\,
 \cos w\,\sin w\\
\Gamma_{132}=&\left(b-a\right)\,\left(b+a\right)\,u\,\cos v\,
 \sin v\,\cos w\,\sin w\\
\Gamma_{133}=&u\,\sin ^2v\,\left(a^2\,\sin ^2w+b^2\,\cos ^2w
 \right)\\
\Gamma_{221}=&-u\,\left(b^2\,\sin ^2v\,\sin ^2w+a^2\,\sin ^2v\,
 \cos ^2w+c^2\,\cos ^2v\right)\\
\Gamma_{222}=&-u^2\,\cos v\,\sin v\,\left(b^2\,\sin ^2w+a^2\,
 \cos ^2w-c^2\right)\\
\Gamma_{223}=&-\left(b-a\right)\,\left(b+a\right)\,u^2\,\sin ^2
 v\,\cos w\,\sin w\\
\Gamma_{231}=&\left(b-a\right)\,\left(b+a\right)\,u\,\cos v\,
 \sin v\,\cos w\,\sin w\\
\Gamma_{232}=&\left(b-a\right)\,\left(b+a\right)\,u^2\,\cos ^2v
 \,\cos w\,\sin w\\
\Gamma_{233}=&u^2\,\cos v\,\sin v\,\left(a^2\,\sin ^2w+b^2\,
 \cos ^2w\right)\\
\Gamma_{331}=&-u\,\sin ^2v\,\left(b^2\,\sin ^2w+a^2\,\cos ^2w
 \right)\\
\Gamma_{332}=&-u^2\,\cos v\,\sin v\,\left(b^2\,\sin ^2w+a^2\,
 \cos ^2w\right)\\
\Gamma_{333}=&-\left(b-a\right)\,\left(b+a\right)\,u^2\,\sin ^2
 v\,\cos w\,\sin w\\
\end{eqnarray*}

You can verify that this seems reasonable by setting $a=b=c=1$ in which case many of these
become zero and we are left with the six symbols from the Spherical coordinates. The formula
for the inverse of the metric tensor is not given as it turns out to be quite big, but using
Maxima we can calculate the Christoffel symbols of the second kind and there turns out to be six
of these just as in the case of the Spherical coordinates.

\begin{twocol}{
\begin{eqnarray*}
\Gamma_{12}^{2}=&{{1}\over{u}}\\
\Gamma_{13}^{3}=&{{1}\over{u}}\\
\Gamma_{22}^{1}=&-u\\
\end{eqnarray*}}{
\begin{eqnarray*}
\Gamma_{23}^{3}=&{{\cos v}\over{\sin v}}\\
\Gamma_{33}^{1}=&-u\,\sin ^2v\\
\Gamma_{33}^{2}=&-\cos v\,\sin v\\
\end{eqnarray*}}
\end{twocol}


\section{Other Coordinates}
This coordinate system is taken from \cite{night} p. 11. It 
probably does not have any practical applications, but it is the first
example of a metric tensor where all components are nonzero.

$$
    \vec{r}(u,v,w) = (u+v, u-v, 2uv + w)
$$

$$
   g_{ij} = \begin{pmatrix}
               4v^2 +2 & 4uv & 2v \\
               4uv & 4u^2 + 2 & 2u \\
               2v & 2u & 1
            \end{pmatrix}
$$

There are only three Christoffel symbols of the first kind
\begin{eqnarray*}
\lcs_{121}=&4v \\
\lcs_{122}=&4u \\
\lcs_{123}=&2
\end{eqnarray*}

The inverse of the metric tensor is

$$
   g^{ij} = \begin{pmatrix}
               1/2  & 0  & -v \\
               0 & 1/2 & -u  \\
               -v & -u & 2v^2 + 2u^2+1 
            \end{pmatrix}
$$
This means that we are left with only one Christoffel symbol of the second kind
\begin{eqnarray*}
\mcs{12}{3}=&2 \\
\end{eqnarray*}

\section{\myindx{Geodesics}}

Why bother with all these Christoffel symbols? The reason is that in order to find
the \myindx{equations of motions} in General Relativity we need to find the possible 
patterns of movements (straight lines) and these are called Geodesics. There is a connection between
Geodesics and the Christoffel symbols given by 

$$
    \frac{d^2u^i}{dt^2} + \Gamma_{jk}^i\frac{du^j}{dt}\frac{du^k}{dt} = 0
$$

Now let us take the Spherical coordinate system and find the Geodesic curves. First by
noting that $u^i=u^1=r$,

\begin{eqnarray}
    \frac{d^2r}{dt^2} + \Gamma_{jk}^1\frac{du^j}{dt}\frac{du^k}{dt} = 0 \\
    \frac{d^2r}{dt^2} + \Gamma_{22}^1(\dot{\theta})^2
    + \Gamma_{33}^1(\dot{\phi})^2 = 0
\end{eqnarray}
here $u^i=u^2=\theta$,
$$
    \frac{d^2\theta}{dt^2} + \Gamma_{jk}^2\frac{du^j}{dt}\frac{du^k}{dt} = 0
$$
$$
    \frac{d^2\theta}{dt^2} + \Gamma_{12}^2\dot{r}\dot{\theta}
    + \Gamma_{33}^2(\dot{\phi})^2 = 0
$$

finally by setting $u^i=u^3=\phi$, we get
\begin{eqnarray*} 
    \frac{d^2\phi}{dt^2} + \Gamma_{jk}^3\frac{du^j}{dt}\frac{du^k}{dt} = 0 \\
    \frac{d^2\phi}{dt^2} + \Gamma_{13}^3\dot{r}\dot{\phi}
    + \Gamma_{23}^3\dot{\theta}\dot{\phi} = 0
\end{eqnarray*}
so the equations of motion are given by the three second-order differential
equations
\begin{eqnarray*} 
    \ddot{r} -r\dot{\theta}^2 -r \cos^2\theta\dot{\phi}^2 = 0\\
    \ddot{\theta} + \frac{1}{r}\dot{r}\dot{\theta}
    + \cos\theta\sin\theta(\dot{\phi})^2 = 0 \\
    \ddot{\phi} + \frac{1}{r}\dot{r}\dot{\phi}
    - \frac{sin\theta}{\cos\theta}\dot{\theta}\dot{\phi} = 0
\end{eqnarray*}
This seems complicated, but let us see what it means in a simple case, where $\theta$ is 
constant at $\theta_0$ and $r=1$. In this case the equations reduce to

\begin{eqnarray}
    -cos^2\theta_0\dot{\phi}^2 = 0\\
    \ddot{\phi}  = 0
\end{eqnarray}

from the first equation we get $\theta_0=\pi/2$, and from the second equation we get
$\phi(t) = at+b$.

This means that the geodesic curves of constant $\theta$ and $r$ are great circles on the 
equator.

\fixme{Find a better (simpler) example?}

\section{Higher dimensional spaces}
\subsection{\myindx{Schwarzchild metric (4d)}}
\label{sec:schwarz}
An important metric in General Relativity is the Schwarzchild metric which is a topic in 
section \ref{sec:4dcurves}. When the four coordinates are $t, r, \theta, \phi$, and 
 $a=GM/c^2$ and spherical coordinates are used, its metric tensor is

$$
g_{ij}= 
\begin{pmatrix}
1-{{a}\over{r}}&0&0&0\cr 
0&-{{1}\over{1-{{a}\over{r}}}}&0& 0\cr 
0&0&-r^2&0\cr 0&0&0&-r^2\sin ^2\theta\cr 
\end{pmatrix}
$$

The Christoffel symbols of the first kind are

\begin{twocol}{
\begin{eqnarray*}
\Gamma_{112}=&{{ar-a^2}\over{2r^3}}\\
\Gamma_{121}=&{{a}\over{2r^2-2ar}}\\
\Gamma_{222}=&-{{a}\over{2r^2-2ar}}\\
\Gamma_{233}=&{{1}\over{r}}\\
\Gamma_{244}=&{{1}\over{r}}\\
\end{eqnarray*}
}{
\begin{eqnarray*}
\Gamma_{332}=&a-r\\
\Gamma_{344}=&{{\cos \theta}\over{\sin \theta}}\\
\Gamma_{442}=&\left(a-r\right)\sin ^2\theta\\
\Gamma_{443}=&-\cos \theta\sin \theta\\
\end{eqnarray*}
}
\end{twocol}

and the inverse metric tensor is 
$$
g^{ij} = 
\begin{pmatrix}
{{1}\over{1-{{a}\over{r}}}}&0&0&0\cr 0&{{a}\over{r}}-1&0&0
 \cr 0&0&-{{1}\over{r^2}}&0\cr 0&0&0&-{{1}\over{r^2\sin ^2\theta
 }}\cr 
\end{pmatrix}
$$

and the Christoffel symbols of the second kind are given as 

\begin{twocol}{
\begin{eqnarray*}
\Gamma_{11}^{2}=&-{{a}\over{2r^2}}\\
\Gamma_{12}^{1}=&{{a}\over{2r^2}}\\
\Gamma_{22}^{2}=&{{a}\over{2\left(1-{{a}\over{r}}\right)^2r^2 }}\\
\Gamma_{23}^{3}=&-r\\
\end{eqnarray*}}
{
\begin{eqnarray*}
\Gamma_{24}^{4}=&-r\sin ^2\theta\\
\Gamma_{33}^{2}=&r\\
\Gamma_{34}^{4}=&-r^2\cos \theta\sin \theta\\
\Gamma_{44}^{2}=&r\sin ^2\theta\\
\Gamma_{44}^{3}=&r^2\cos \theta\sin \theta\\
\end{eqnarray*}}
\end{twocol}

\subsection{\myindx{Kaluza-Klein metric (5d)}}
A space of five dimensions were contrieved in a unification theory by Kaluza and Klein.
Several variations of the theme exist, but the following is an example from \cite{art:tiwari} where
the  metric tensor is


$$
 \hat{g}_{\hat{\mu}\hat{\nu}} = 
     \begin{pmatrix}
         1  &  0  &  0   &  0                 & 0 \\
         0  & -A^2(t)  &  0   &  0                 & 0 \\
         0  &  0  & -A^2(t)r^2 &  0                 & 0 \\
         0  &  0  &  0   & -A^2(t)r^2\sin^2\theta & 0 \\
         0  &  0  &  0   &  0                 & -B^2(t) 
    \end{pmatrix} 
$$
the inverse of the metric tensor is 
$$
 \hat{g}^{\hat{\mu}\hat{\nu}} = 
     \begin{pmatrix}
         1  &  0  &  0   &  0                 & 0 \\
         0  & -\frac{1}{A^2(t)}  &  0   &  0                 & 0 \\
         0  &  0  & -\frac{1}{A^2(t)r^2} &  0                 & 0 \\
         0  &  0  &  0   & -\frac{1}{A^2(t)r^2\sin^2\theta} & 0 \\
         0  &  0  &  0   &  0                 & -\frac{1}{B^2(t)} 
    \end{pmatrix} 
$$
and the Christoffel symbols of the second kind are



\begin{twocol}{
\begin{eqnarray*}
\Gamma_{12}^{2}=&\frac{\dot{A}}{A}\\
\Gamma_{13}^{3}=&\frac{\dot{A}}{A}\\
\Gamma_{14}^{4}=&\frac{\dot{A}}{A}\\
\Gamma_{15}^{5}=&\frac{\dot{B}}{B}\\
\Gamma_{22}^{1}=&A\dot{A}\\
\Gamma_{23}^{3}=&\frac{1}{r}\\
\Gamma_{24}^{4}=&\frac{1}{r}
\end{eqnarray*}
}{
\begin{eqnarray*}
\Gamma_{33}^{1}=&r^2A\dot{A}\\
\Gamma_{33}^{2}=&-r\\
\Gamma_{34}^{4}=&\frac{\cos\theta}{\sin\theta}\\
\Gamma_{44}^{1}=&r^2\sin^2\theta A\dot{A}\\
\Gamma_{44}^{2}=&-r\sin^2\theta\\
\Gamma_{44}^{3}=&-\cos\theta\sin\theta\\
\Gamma_{55}^{1}=&B\dot{B}
\end{eqnarray*}}
\end{twocol}
which is not too bad considering that there could have been $5^3=125$!. Note that the symbols depend 
on the choice of the functions $A(t)$ and $B(t)$ and their
derivatives. The functions would have to be found as a solution to the Einstein field equations,
but one could assume a form of the functions and then calculate the consequences for 
\myindx{cosmology}.

