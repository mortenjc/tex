\section{Curves in higher dimensions}

Some of the previous discussions of parametric functions can be 
generalized to higher dimensions. For example the trajectory of 
an object as it travels through 3-dimensional space can be
represented as the position of its center of mass as a function 
of time

$$
   (x,y,z) = \big(X(t), Y(t), Z(t)\big)
$$
whose tangent vector is 
$$
    \vec{v}_t = \big(X'(t), Y'(t), Z'(t)\big)
$$
or if several physical values are measured for a process as a function of time, these 
can be viewed as a parametric curve in n-dimenstions

$$
   \vec{p}(t) = \big(x_1(t), x_2(t), \ldots, x_n(t)\big)
$$
where we have replaced the $x,y,z$ coordinates by $x_1,\ldots,x_n$.

\index{length of!parametric curve}
\index{length of!n-dimensional curve}
A straight forward extension of equation \ref{eq:length} for the length of a parametric curve in 
2 dimensions gives, the length of an n-dimensional curve  

$$
  \mathcal{L} =  \int_a^b\sqrt{\dot{x}_1^2 + \dot{x}_2^2 + \ldots + \dot{x}_n^2 } dt
$$


