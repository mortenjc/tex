\newcommand{\mypeople}[4]{
\index{#1}
\noindent{ \textbf{#1} (#2):\emph{#3}. #4
}\vspace{0.2cm}}
\chapter{People of Geometry}

This is just a brief collection of names of people who have been
influential in either pure geometry or the geometry of spacetime.\\




\mypeople{Pythagoras}{570-495 BC}{Pythagoras' theorem}
{Attributed as the discoverer of Pythagora's theorem relating 
the side of a right angled triangle to its diagonal. Was influential
in creating a natural philosophical school whose ideas has 
reverbated for millennia.}

\mypeople{Euclid}{ca. 300 BC}{Geometry and number theory}
{Also named the father of geometry, Euclud wrote a treatise "Elements" on 
geometry that kept inspiring mathematicians for two millennia. Euclids
geometry was eventually extended to curved spaces.}

\mypeople{Nicolaus Copernicus}{1473-1543}{Astronomy, heliocentric model}
{Copernicus was a renaissance astronomer who challenged the current
view that earth was the center of the universe by suggesting that a 
heliocentric model better fitted observation and simplified the
apparent movements of the planets in the night sky. }

\mypeople{Galileo Galilei}{1564-1642}{Investigation of motion}
{Galileo has been called the \myindx{father of modern science} due to his 
introduction of systematic investigations. He performed a number of 
scientific studies of motion by
rolling balls on an inclined plane and concluded that 
objects of different mass or density would fall at the same speed were it not for 
wind resistance, friction etc. A test of his claim was actually performed 
on the moon by one of the \myindx{Apollo} missions.Galileo built 
a telescope and was the first to observe Saturns rings.}

\mypeople{Johannes Kepler}{1571-1630}{Laws of planetary motion}
{Kepler was the first to describe the laws of planetary motion based on 
observational astronomy. Among Keplers other contributions were a conjecture 
about the optimal packing of spheres which were only recently proved -
nearly 400 years after the conjecture was made.}

\mypeople{Isaac Newton}{1643-1727}{Invention of differential calculus, equations of motion}
{Newton was in many
respects the father of celestial mechanics. He was the first to formulate the 
principles of gravitational attraction and from this deduce the equations
of motions for planets orbiting the sun. In the process he also invented 
differential calculus. Newtons physics were unchallenged for nearly 300 years
when they were extended by Einstein. For many practical purposes - even
putting a man on the moon, Newtons formulae are still adequate. Newtons
main work is the \myindx{Principia} which deals with optics, mathematics and physics.}

\mypeople{Leonhard Euler}{1707-1783}{Mathematical foundations and geometry}
{Euler developed much of the modern symbols used in physics and contributed
in every field of pure and applied mathematics. Euler was by any standards
the most productive mathematician the world has produced. For an excellent book
on some of his contributions put in historical perspective see \cite{dunham}.}

\mypeople{Karl Friedrich Gauss}{1777-1855}{Curvature and geometric measurements}
{Gauss made numerous contributions to pure and applied mathematics. He developed
methods for accurate surveying and investigated the curvature of surfaces. He 
discovered that curvature of a surface is an intrinsic property of the surface,
and thus can be measured on the surface and provided a formula for 
calculating the curvature of a surface from the metric tensor. Gauss is 
sometimes referred to as the \myindx{prince of mathematics}.  
The book \cite{gauss} is a very read-worthy fiction of Gauss' life and carreer.}

\mypeople{Carl Gustav Jacob Jacobi}{1804-1851}{Transformation matrix}
{Jacobi was a matematician and contributed much to the field. In relation
to General Relativity it is the \myindx{Jacobian matrix} involved in coordinate
transforms that are the most significant.}

\mypeople{Bernhard Riemann}{1826-1866}{Definition of Manifolds}
{Riemann extended Gauss' work on differential Geometry to higher
dimensions, and founded the field of Riemannian Geometry. The Riemann curvature 
tensor is names after him. Perhaps his most well known contribution
is to analytic number theory, where \myindx{Riemanns zeta function} was 
introduced and his famous conjecture was made.}

\mypeople{Elwin Bruno Christoffel}{1829-1900}{Tensor algebra}
{Christoffel contributed to the development of Tensor algebra
which is so fundamental to General Relativity. The Christoffel
symbols named after him are essential in tensor analysis.}

\mypeople{Gregorio Ricci-Curbastro}{1853-1925}{Tensor algebra}
{Riccis contribitions were in the fields of geometry, dynamics
and tensor analysis.}

\mypeople{Hermann Minkowsky}{1864-1909}{Spacetime}
{Minkowsky made the connection between time and space, combining
them in what is now called \myindx{spacetime}. This was one of the steps
in the direction of General Relativity.}

\mypeople{Karl Schwarzchild}{1873-1916}{Solution to the EFE}
{Shortly after Einsteins new theory was presented, Schwarzchild
solved the equations for empty space outside a single nonrotating, chargless mass.
This achievement was made while Schwarzchild was a soldier in the trenches 
of World War I.}

\mypeople{Albert Einstein}{1879-1955}{General Relativity}
{Einstein made four big contributions to modern physics, but the most
profound and intellectual was the formulation of the equations of motion
extending Newtons formula to massive objects and high speeds. The 
discovery, or rather the construction of the formula was a monumental
task of mathematics. Einsteins theory is applied in modern GPS receivers.}

\mypeople{Roy Kerr}{1934-}{Solution to the EFE}
{About 45 years after the first exact solution to the Einstein Field Equations (EFE)
Kerr found a solution for a rotating mass.}

\mypeople{Kip Thorne}{1940-}{Authority on General Relativity}
{Thorne co-authored the textbook on Gravitation that is still the most
comprehensive textbooks on the subject. Thorne has been involved 
in validation of Einsteins theory by the gravitational wave experiment LIGO.}


