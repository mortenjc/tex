\chapter{Curves in higher dimensions}
\label{sec:4dcurves}

There is no reason to stop at 3D as mathematics do not
place any limits on curves.

In fact General Relativity operates with the notion of \myindx{spacetime} which 
is a four dimensional construction, and in other areas of physics
even higher dimensions are used. \myindx{String theory} which tries to 
unite the four fundmental forces operates with dimensions of 10 and
upwards. Moving to higher dimensions, however, presents some difficulties as 
1) the metric tensor has higher dimensionality and hence more terms and 
cross terms to evaluate. 2) For dimensions higher than 3 graphical visualization
is impossible.


\section{Spacetime}
Spacetime is the incorporation of time as a fourth coordinate into three dimensional space.
$$
    u_\nu = (x_0, x_1, x_2, x_3) = (ct, -x, -y, -z)
$$
With the metric tensor
\index{metric tensor!spacetime}
$$
   g_{\mu\nu} = \begin{pmatrix}
       c &  0 &  0 &  0 \\ 
       0 & -1 &  0 &  0 \\ 
       0 &  0 & -1 &  0 \\ 
       0 &  0 &  0 & -1 \\ 
   \end{pmatrix} 
$$
and a the distance between nearby points is $\Delta s^2 = c^2dt^2 - dx^2 - dy^2 - dz^2$.

Note that since all elements of $g$ are constant, all Christoffel symbols are $0$, this space
is flat!


\section{\myindx{Schwarzchild Metric}}
The Schwarzchild metric is a specific solution to Einsteins field equations for gravitation
around a massive object in empty space given spherical symmetry. Examples of the derivation
of this metric are given in \cite{night} and \cite{padman}.

\index{metric tensor!Schwarzchild}
$$
   g_{\mu\nu} = \begin{pmatrix}
        1 - 2MG/c^2r & 0                  &   0  & 0     \\
        0            & -(1-2MG/c^2r)^{-1} &   0  & 0     \\
        0            &                    & -r^2 & 0     \\
        0            &         0          &   0  & -r^2\sin^2\theta 
      \end{pmatrix}
$$
the Christoffel symbols for this metric are given in section \ref{sec:schwarz}.


\begin{figure}
\begin{center}
\psset{unit=0.75} 
\begin{pspicture}(-5.5,-7)(4.5,4) 
  \psset[pst-solides3d]{viewpoint=20 120 30 rtp2xyz, Decran=50,lightsrc=10 125 20 rtp2xyz}
  % Parametric Surfaces
  \defFunction{hole}(u,v) 
      {u} 
      {v} 
      { 1 neg u dup mul v dup mul add 0.2 add div}
  \defFunction{curve}(t) 
      {t Cos 3 mul 1.5 sub} 
      {t Sin 3 mul 1.5 add} 
%      { 0}
      { 1 neg t Cos 3 mul 1.5 sub dup mul t Sin 3 mul 1.5 add dup mul add 0.2 add div}
  \psSolid[object=surfaceparametree, linecolor=black!70, base=3 neg 3 3 neg 3, 
         fillcolor=blue!50,incolor=black!90, function=hole,linewidth=0.5\pslinewidth,ngrid =30]% 
      \psSolid[object=courbe, r=0, range= 0.5 2.05 neg , action=draw*, linecolor=black, 
               linewidth=0.05,resolution=360, function=curve]
\end{pspicture}
\end{center}
\vspace{1cm}
\begin{center}
 \caption{\small Curvature of space near a massive object.}
\end{center}
\end{figure}


\section{\myindx{Kerr Metric}}
For a rotating massive object the metric tensor is a further generalization of the Schwarzchild 
metrix, where

\index{metric tensor!Kerr}
$$
   g_{\mu\nu} = \begin{pmatrix}
      \left( 1 - \frac{r_sr}{\rho^2}c^2 \right) & 0 & 0 & \frac{r_sr\alpha\sin^2\theta}{\rho^2} \\
      0         & - \frac{\rho^2}{\Delta} & 0 &  0 \\
      0 & 0 & -\rho^2 & 0 \\ 
      \frac{r_sr\alpha\sin^2\theta}{\rho^2} & 0 & 0 & \left( r^2 + \alpha^2 + \frac{r_sr\alpha^2}{\rho^2}\sin^2\theta\right)\sin^2\theta 
      \end{pmatrix}
$$
where 

\begin{twocol}{
\begin{eqnarray*}
   r_s =& \frac{2GM}{c^2} \\
   \alpha =& \frac{J}{Mc} \\
\end{eqnarray*}}{
\begin{eqnarray*}
   \rho^2 =& r^2 + \alpha^2\cos^2\theta \\
   \Delta =& r^2 -r_sr + \alpha^2 
\end{eqnarray*}}
\end{twocol}

In the limit of no rotation (where $J=0$) the metric reduces to the Schwarzchild metric.

\section{\myindx{Robertson-Walker Metric}}
\myindx{Friedmann}
\myindx{Lema\^itre}
\myindx{Einstein field equations}

Or Friedmann-Lema\^itre-Robertson-Walker metric as it probably should be named due to its 
many contributors.  This is another exact solution of the Einstein Field equations in the context 
\myindx{cosmology}, where a homogenous and isotropic, expanding (or contracting) universe is 
described. The metric tensor is given as 

\index{metric tensor!Robertson-Walker}
$$
 g_{\mu\nu} =   \begin{pmatrix}
    c^2 & 0 & 0 & 0 \\
    0 & -A(t)^2\frac{1}{1-kr^2} & 0 & 0 \\
    0 & 0 & -A(t)^2r^2 & 0 \\
    0 & 0 &  0 & -A(t)^2r^2\sin^2\theta
    \end{pmatrix}
$$
where $A(t)$ is a \myindx{scale factor} affecting the spatial components.

$$
 g^{\mu\nu} =   \begin{pmatrix}
   \frac{1}{c^2} & 0 & 0 & 0\\
   0 & -\frac{1+kr^2}{A^2} & 0 & 0 \\
   0 & 0 & -\frac{1}{r^2A^2} & 0 \\
   0 & 0 & 0 & -\frac{1}{A^2r^2\sin^2\theta} 
  \end{pmatrix}
$$
and 

\begin{twocol}{
\begin{eqnarray*}
   \Gamma_{12}^2 = \frac{\dot{A}}{A} \\
   \Gamma_{13}^3 = \frac{\dot{A}}{A} \\
   \Gamma_{14}^4 = \frac{\dot{A}}{A} \\
   \Gamma_{22}^1 = \frac{A\dot{A}}{c^2(1+kr^2)} \\
   \Gamma_{22}^2 = -\frac{kr}{1+kr^2} \\
   \Gamma_{23}^3 = \frac{1}{r} \\
\end{eqnarray*}}{
\begin{eqnarray*}
   \Gamma_{24}^4 = \frac{1}{r} \\
   \Gamma_{33}^1 = \frac{r^2A\dot{A}}{c^2} \\
   \Gamma_{33}^2 = -r(1+kr^2) \\
   \Gamma_{34}^4 = \frac{\cos\theta}{\sin\theta} \\
   \Gamma_{44}^1 = \frac{r^2A\dot{A}\sin^2\theta}{c^2} \\
\end{eqnarray*}}
\end{twocol}

