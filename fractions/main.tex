\documentclass{article}
\usepackage[danish,english]{babel}
\usepackage{inputenc}
\usepackage[table]{xcolor}
\usepackage{colortbl}
\usepackage{amsmath,amsthm,booktabs}
\usepackage[pdftex]{graphicx}
\usepackage{tikz}
\usepackage{pgfplots}
\usetikzlibrary{calc,through,decorations.fractals,plotmarks}

\newcommand\QED{$\mathcal{Q.E.D}$}
\newcommand\mytablecolor{\rowcolors{1}{pink}{white}}


\newenvironment{twocol}[2]{%
\begin{minipage}{0.5\linewidth}
#1
\end{minipage}
\begin{minipage}{0.5\linewidth}
#2
\end{minipage}
}
{}

\newtheorem{theorem}{Theorem}

\begin{document}

\section{The approximation of algebraic irrational numbers and Liouville's transcendental numbers}
Suppose that

\begin{equation}
     f(x) = a_0 + a_1x + \cdots + a_nx^n   \label{polyn}
\end{equation}
is a polynomial of degree $n$ with {\em integral} coefficients $a_0, a_1, \cdots, a_n$. Then,
a root, $\alpha$, of this polynomial is said to be {\em algebraic}. Since every rational
number $\alpha = a/b$ can be defined as the root of the first-degree equation
$bx-a = 0$, the concept of an algebraic number is clearly a natural generalization of the concept of 
a rational number. If a given algebraic number satifies an equation $f(x)=0$ of degree $n$, and does
not satisfy any equation of lower degree (with integral coefficients), it is called an algebraic number of degree
$n$. In particular, rational numbers can be defined as first-degree algebraic numbers. The number
$\sqrt{2}$, being a root of the polynomial $x^2 -2$, is a second-degree algebraic number, or, 
as we say, a quadratic irrational. Cubic, fourth-degree, and higher irrationals are defined analogously. All
non-algebraic numbers are said to be {\em transcendental}. Examples of transcendental numbers
are $e$ and $\pi$. Because of the great role that algebraic numbers play in contemporary number theory,
many special studies have been devoted to the question of their properties with regard to their
approximation by rational fractions. The first noteworthy result in this direction was the following theorem, known as Liouville's
theorem.

\begin{theorem} \label{liouville}
For every real irrational algebraic number $\alpha$ of degree $n$,
there exists a positive number $C$ such that, for arbitrary integers $p$ and $q \; (q>0)$,

\begin{equation*}
   \left| \alpha - \frac{p}{q} \right| >  \frac{C}{q^n}
\end{equation*}
\end{theorem}

\begin{proof}
Suppose that $\alpha$ is a root of the polynomial (\ref{polyn}). From algebra,
we may write
\begin{equation}
     f(x) = (x- \alpha)f_1(x),   \label{rootoff}
\end{equation}
where $f_1(x)$ is a polynomial of degree $n-1$. Here,  $f_1(\alpha) \ne 0$. To show this, suppose that
$f_1(\alpha) = 0$. Then, the polynomial $f_1(x)$ could be
divided (without a reminder) by $x-\alpha$ and, hence, the polynomial $f(x)$ could be divided by
$(x-\alpha)^2$. But, then, the derivative $f'(x)$ could be divided by $x-\alpha$; that is, we would have $f'(\alpha) = 0$, 
which is impossible since $f'(x)$ is a polynomial of degree $n-1$ with integral 
coefficients and $\alpha$ is an algebraic number of degree $n$. Hence, $f_1(\alpha) \ne 0$,
and, consequently, we can find a positive number $\delta$ such that

\begin{equation*}
    f_1(x) \ne 0   \qquad \qquad  (\alpha - \delta \leq  x \leq \alpha + \delta).
\end{equation*}

Suppose that $p$ and $q \; (q>0)$, are an arbitrary pair of integers. If
$|\alpha -(p/q)| \leq \delta$, then $f_1(p/q) \neq 0$, and, by substituting $x= p/q$ in identity (\ref{rootoff}), we obtain

\begin{eqnarray*}
    \frac{p}{q} - \alpha  & = & \frac{f\left(\frac{p}{q} \right)  }   { f_1\left( \frac{p}{q} \right)     } =
   \frac {
       a_0 + a_1 \left( \frac{p}{q}\right) + \cdots  +  a_n \left( \frac{p}{q}\right)^n
   }{
       f_1\left( \frac{p}{q} \right)   
   }  \\
  & = & \frac{
                a_0q^n + a_1pq^{n-1} + \cdots + a_np^n
            }{
                q^nf_1\left( \frac{p}{q}\right)
            }.
\end{eqnarray*}

The numerator of this fraction is an integer. It is also non-zero, because otherwise we would have $\alpha = p/q$, whereas
$\alpha$ is by hypothesis irrational. Consequently, this numerator is at least equal to unity in absolute value. We denote
by $M$ the least upper bound of the function $f_1(x)$ in the interval $(\alpha -\delta, \alpha + \delta)$. From the last 
inequality, we thus obtain

\begin{equation*}
   \left | \alpha - \frac{p}{q} \right | \ge  \frac{1}{Mq^n}.
\end{equation*}

In the event that 
\begin{equation*}
   \left | \alpha - \frac{p}{q} \right| > \delta ,
\end{equation*}
it follows that 

\begin{equation*}
   \left| \alpha - \frac{p}{q} \right| >  \frac{\delta}{q^n} ,
\end{equation*}
and if we now denote by $C$ any positive number less than $\delta$ and $1/M$,
we obtain, in both cases (that is, for arbitrary $q>0$ and $p$),

\begin{equation*}
   \left| \alpha - \frac{p}{q} \right| >  \frac{C}{q^n},
\end{equation*}
which completes the proof of Theorem \ref{liouville}.

\end{proof}



\end{document}
