\chapter{Curvature}

In this chapter we will give the formulae for working with
curvatures. This part is filled with symbols which makes it somewhat hard
to comprehend. Just consider the symbols as functions which can be
calculated from the metric tensor. 

\index{curvature!Gaussian}
\section{\myindx{Gaussian Curvature}}
Gauss' curvature formula describes the curvature of a two dimensional
surface parametrized by $(u,v)$.


$$
K = -\frac{1}{g_{11}} \left( \frac{\partial}{\partial u}\Gamma_{12}^2 - \frac{\partial}{\partial v}\Gamma_{11}^2 + \Gamma_{12}^1\Gamma_{11}^2 - \Gamma_{11}^1\Gamma_{12}^2 + \Gamma_{12}^2\Gamma_{12}^2 - \Gamma_{11}^2\Gamma_{22}^2\right)
$$

\section{Riemann curvature tensor}
\index{Riemann tensor} 
\index{curvature!Riemann}

The Riemann curvature tensor is a tensor of the fourth order, which 
is defined in the following way
$$
{R^\rho}_{\sigma\mu\nu} = \partial_\mu\Gamma^\rho_{\nu\sigma}
    - \partial_\nu\Gamma^\rho_{\mu\sigma}
    + \Gamma^\rho_{\mu\lambda}\Gamma^\lambda_{\nu\sigma}
    - \Gamma^\rho_{\nu\lambda}\Gamma^\lambda_{\mu\sigma}
$$

\section{Ricci curvature tensor}
\index{curvature!Ricci}
The \myindx{Ricci tensor} is related to the Riemann curvature 
tensor by contracting two indices
$$
   R_{\sigma\nu} = {R^\rho}_{\sigma\rho\nu} =
{\Gamma^\rho_{\nu\sigma}}_{,\rho} - \Gamma^\rho_{\rho\sigma ,\nu}
+ \Gamma^\rho_{\rho\lambda} \Gamma^\lambda_{\nu\sigma}
- \Gamma^\rho_{\nu\lambda}\Gamma^\lambda_{\rho\sigma}
$$


\myhrule
\begin{myex}
For the spherical surface with unit length the only nonzero Christoffel
symbols are
\begin{eqnarray*}
   \Gamma^2_{12} &=& \;\;\frac{\cos\theta}{\sin\theta} \\
   \Gamma^1_{22} &=& -\cos\theta\sin\theta
\end{eqnarray*}
of which only $\Gamma^2_{12}$ appears in Gauss' formula.
Noting that $g_{11}=1$, the Gaussian curvature is then

$$
    K = -\frac{d}{d\theta}\frac{\cos\theta}{\sin\theta} - \left( \frac{\cos\theta}{\sin\theta} \right)^2 = 1
$$
which shows that the sphere has a constant, positive curvature.
\end{myex}


\begin{myex}
A for the hyperbolic paraboloid, the  nonzero Christoffel symbols are
\begin{eqnarray*}
   \Gamma_{11}^1 &=&  \;\;\frac{4u}{4v^2+4u^2+1} \\ 
   \Gamma_{11}^2 &=& -\frac{4v}{4v^2+4u^2+1} \\ 
   \Gamma_{22}^1 &=& -\frac{4v}{4v^2+4u^2+1} \\ 
   \Gamma_{22}^2 &=& \;\;\frac{4v}{4v^2+4u^2+1} 
\end{eqnarray*}
the gaussian formula reduces to
$$
   K = \frac{1}{g_{11}} \left(\frac{d}{dv}\Gamma_{11}^2 + \Gamma_{11}^2\Gamma_{22}^2 \right)
     = -\frac{4}{(4v^2+4u^2+1)^2}
$$
The Gaussian curvature is now a function of $u$, and $v$,
and we see that this surface has negative curvature everywhere with a minimum
occuring at $(u,v)=(0,0)$, where it has the value $-4$.
\end{myex}

