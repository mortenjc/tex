\chapter{Introduction}

æøåÆØÅ

I began writing this document as a side effect of trying to understand General Relativity, which makes 
heavy use of \myindx{differential geometry}.

Here I will show some examples of 2D surfaces embedded in 3D euclidian spaces, how to create 
curves on these surfaces and how to evaulate the length of these curves. The necessary 
mathematics is introduced but the reader is assumed to be familiar with vectors, matrices and 
\myindx{differential calculus}. Several examples are given with detailed calculations.


Chapter \ref{sec:vectors} gives some basic definitions of vectors and their properties. 
We then do a quick brush up on functions of one variable in 2-dimensions in chapter \ref{sec:2dcurves}, 
how to find the areas and lengths of these curves, etc. 
Chapter \ref{sec:coordsys} describes a number of different coordinate systems in 3-dimentional 
space and how to transform from one coordinate system to another. Following this is a short 
description of maps and projections in chapter \ref{maps}.
In chapter \ref{sec:surfaces} we show how a surface can be described as a parametrization of the 
three dimentional coordinates $\vec{r}$ by two independent variables, and how curves can be 
described as the parametrization of $\vec{r}$ by one variable. In section \ref{sec:christoffel} a 
number of detailed examples of calculations based on the metric
tensor are given.
A brief description of parallel transport of vectors along curves is given in chapter \ref{sec:parallel}. 
Chapter \ref{sec:4dcurves} extends the previous ideas to higher dimensions.

\vspace{0.5cm}
\index{latex package!amsmath}
\index{latex package!graphicx}
\index{latex package!listings}
\index{latex package!amsthm}
\index{latex package!fancyhdr}
\index{latex package!pstricks}
\index{latex package!pst-3dplot}
The document is typeset with \LaTeX a lot of different packages: \emph{\myindx{amsmath}, \myindx{graphicx}, 
\myindx{listings}, \myindx{amsthm},
\myindx{fancyhdr}},...the list goes on. The figures are created by tools with varying support for 
integration with \LaTeX\ such as \textbf{gnuplot}, \TikZ\, \emph{\myindx{pstricks}} and \emph{\myindx{pst-3dplot}} packages. 
Where numerical results are required, such as difficult 
integrals, or simplification of formulae \url{\myindx{WolframAlpha.com}} has been used as 
well as Romberg integration 
routines written in C and C++. To reduce the occurence of manual errors, the software package Maxima 
was used for symbolic manipulation of algebraic expressions. Appendix \ref{sec:tools} gives details
on how to use these tools in conjunction with \LaTeX.


\begin{figure}
  \begin{center}
  \psset{unit=0.75} 
  \begin{pspicture}(-5.5,-7)(4.5,4) 
    \psset[pst-solides3d]{viewpoint=20 120 30 rtp2xyz, Decran=50,lightsrc=-10 15 10}
    \defFunction{shell}(u,v) 
        {1.2 v exp u Sin dup mul v   Cos mul mul} 
        {1.2 v exp u Sin dup mul v   Sin mul mul} 
        {1.2 v exp u Sin u   Cos mul mul}
    \psSolid[object=surfaceparametree, linecolor=black!70, base=0 pi pi 4 div neg 5 pi mul 2 div, 
         hue=0 1, fillcolor=blue!50,incolor=black!90, function=shell,linewidth=0.5\pslinewidth,ngrid =40]% 
  \end{pspicture}
\end{center}
\vspace{1cm}
\begin{center}
Mathematical surface of a \myindx{logaritmic spiral} very similar to the one created by the \emph{\myindx{Nautilus pompilius}} cephalopod.
\end{center}
\end{figure}
