\section{Introduction}
\IEEEPARstart{P}{acket} data networks are concerned with the forwarding of data packets.
The demand for bandwidth is constantly growing,  and because of advances in technology
we have witnessed packet data networks go from 1Mbit/s to 100Gbit/s (in the 
case of Ethernet) from about 1980 to the present day.
Vendors strive for what is called wire-speed forwarding, which means that they support
forwarding descisions at the incoming packet rates. This aim pushes the implementations of forwarding 
algorithms into Application Specific Integrated Circuits (ASICs).
Common to many packet forwarding technologies are a central switch fabric, and 
a table containing information about known transmitters in the network.
The population of this table is often, but not necessarily always, done by means of hashing. Consequently
the properties of hashing in data packet networks has attracted some attention. For example
\cite{Tiwari2010} gives a survey of cryptographic hash functions, where \cite{Sedgewick2011}  provides 
discussions on the properties on simpler hash functions. One study \cite{Huntley2006} examines the effect of hash collisions 
on the performance of packet forwarding based on  combinatorical calculations and real-world testing, and
\cite{Stancu2003} examines the effective size of switching tables based on testing scenarios, where 
they have full control of the network topology and configuration. Hash tables may not 
even be central \cite{Ray2007}, but distributed across one or more chips.

Once  the ASIC implementation of a hash table is done, there is no possibility of subsequent 
modifications, so any design guidance that can be used in advance should be taken into
account. 
In this paper we will elaborate on both the theoretical aspects of hash collisions in a Ethernet switching 
context. 
This is done by providing a general model of a switch MAC address table, at least one often used 
hashing algorithm in switch ASICS as well as simulations. The hashing methods used in 
wirespeed switches is rarely announced, and the best utilization is often reached in 
unrealistic scenarios~\cite{Huntley2006}\cite{Stancu2003}.

The following discussion will focus on Ethernet networks as a theme, 
mostly because is has proven a very successful and adaptable technology\cite{Song2002},
but the principles are completely general. 
We use the terms bridging and switching interchangable.


\section{Bridging}
\IEEEPARstart{A}{n} Ethernet network consists of one or more interconnected bridges
in a loop-free topology (using the Spanning Tree algorithm). In 
addition to other switches nodes (or end hosts) are also connected.
Each node is uniquely identified by its MAC address, which is a 48 bit 
serial number. The bridges ensures the delivery of packets to end-hosts
in a given segment. For unicast traffic the end-host is normally the only recepient 
of that packet. The forwarding mechanism is based on MAC address
tables holding enough information to make a forwarding descision,
where the switch ... at most one port... \cite{IEEE802.1D} for a detailed description of the MAC address learning and 
forwarding mechanisms see \cite{Ray2007}.

For this purpose, each bridge maintains a forwarding table with entries
of the form $<$MAC address, DPORT$>$. 
Populating the forwarding table is carried out based on the
{\em source addresses} of packets that a bridge receives. 

When subsequently receiving a packet on port P1 with an unknown
source address SA, the bridge then creates (learns) an entry of the form
$<$SA,P1$>$.

The 'learned' entries are then used to make forwarding
decisions for packets {\em destined} for address SA. These
will be sent out on port P1. Now if a packet arrives on 
any port, with a destination address DA which is not in the 
forwarding table, that packet is forwarded on all active ports.
This is also called flooding. Excessive flooding is a potentially big 
problem for the stability and bandwidth utilization in a network.

In the following we will describe a typical implementation of 
forwarding tables and the ...

While this analysis share some similarities with other studies,
these mostly seem to model traffic scenarios, where we look
at ...

