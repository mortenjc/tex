\section{bins and buckets}
\IEEEPARstart{A}{} 
typical implementation of a MAC address forwarding table is 
to make an array of $W$ buckets or bins each able to hold a number
of MAC address entries. We call this number the depth $d$ of the table.

A MAC address table is often specifiied in terms of its size (in entries), but an ASIC designer is probably more concerned with 
how many gates it occupies, the latter being proportional to the number of bits. Given the switch parameters $(n,d)$ the size 
is simply $n\cdot d$. If the size (in bits) of each entry is $s$, the  number of gates are $\approx n\cdot d \cdot s$.



\begin{table}[!t]
\renewcommand{\arraystretch}{1.3}
\caption{ A MAC address table consists of a MAC address, the switch port number where the MAC address was learned
and an age field. The index is implicit but shown here to illustrate its relevance.}
\centering
\begin{tabular}{lrrr}
\hline
\bfseries mac addr. & \bfseries port & \bfseries age & \bfseries index\\ 
\hline
00:23:14:00:00:01 &  3 & 125  & 47 \\
20:00:01:01:02:03 &  17  &  50 & 1 \\
$\cdots$ & $\cdots$ & $\cdots$ & $\cdots$ \\
64:b9:e8:bc:87:1c &  4 & 85   & 47 \\
\hline
\end{tabular}

\end{table}




Mapping a MAC address to a specific bin is typically done by applying a hashing 
function, $h$,  to the $M$ bit MAC address and associated information,  into the $W$ bins.


A hash function, h,  is a surjective function from a domain $\{0,\ldots,2^{n -1}\}$ of addresses to a domain $\{0,\ldots,2^{ m-1}\}$ of indices, 
where $m < n$. Borrowing a notation from cryptography, $h$ maps a bit string of length $n$ into a bit string of length $m$.

$$
   h: \{01\}^m \rightarrow \{01\}^{n}
$$

One popular method is to divide the MAC address into 4,6,8 or 12 bit chunks and then 
XOR these together. The final value identifies the bucket in which
the MAC address should be stored. Later we will discuss some specific hashing functions.

Since the final hash value has many fewer bits than the original data, it will clearly be the case
that multiple MAC addresses will map to the same hash value. This is called a hash collision.


