
\chapter{2D Curves}
\label{sec:2dcurves}

In this chapter we will start by showing the familiar functions of one variable $y=f(x)$. We will quickly
find out that these are somewhat limited in the variety of shapes they can express, 
which leads onwards to the more general parametric 2D curves.

\section{Functions of one variable}
\label{sec:simplefuncs}

A function of one variable, $y=f(x)$ simply connects pairs of numbers: For each $x$ there is a corresponding $y$.
Functions of one variable are suitable for representing simple relations where the $x$ values
are increasing continuously (time, days, months, ...). This means that we cannot use this method to draw
geometry such as circles, ellipses and other closed or self intersecting shapes shown in 
figures \ref{fig-circle2d} and \ref{fig-intersect}.

\begin{figure}[h]
\begin{tikzpicture}[scale=1.5]
   \txfig{0}{0}{3.25}{2.25}
   \draw[domain=0:2.5,smooth,red] plot[parametric,prefix=pgffigs/mjc,id=line] function{t,t};
   \draw[domain=0:2.5,smooth,blue] plot[parametric,prefix=pgffigs/mjc,id=parabola] function{t,(t-1)**2};
   \node [left,red]  at (2.0,2.15) {\fsz $y=x$};
   \node [left,blue] at (3.5,0.75) {\fsz $y=(x-1)^2$};
\end{tikzpicture}
\caption{\small Two simple functions of the single variable $x$.}
\label{fig-simple2d}
\end{figure}

\index{length of!a curve}
The length of a curve specified as y=f(x) is

$$
  \int_a^b\sqrt{(dx)^2 + (\frac{\partial y}{\partial x}dx)^2}=\int_a^b\sqrt{1+(y')^2}dx
$$

The formula can be understood in terms of \myindx{Pythagoras formula} for small changes in $x$ and $y$:  If  
we increment $x$ by $dx$ from $x_0$ to $x_0 + dx$ the corresponding change in $y$, $dy$ is the rate
that $y$ changes with $x$, $\partial y/\partial x$ evaluated at $x_0$,
multiplied with $dx$. The total length is then the \myindx{hypotenuse}
of the right angle triangle with sides $dx$ and $dy$, which is $\sqrt{(dx)^2 + (dy)^2}$. This is 
illustrated in figure \ref{fig-delta_l}.
Integrating this formula over the interval of interest gives the total length of the 
curve.


\begin{figure}[h]
\begin{center}
\begin{tikzpicture}[scale=5]
   \txfig{1.25}{.5}{2.5}{0.75}
   \draw[domain=1:2.25,line width=1.5pt,smooth,blue] plot[parametric,prefix=pgffigs/mjc,id=parabola2] function{t,(t-1)**2};
   \node [left] at (1.5,1.0) {\fsz $y=(x-1)^2$};
   \node [left] at (1.5,.8) {\fsz $\frac{\partial y}{\partial x}=2(x-1)$};
   \draw [black] (1.5,0.25) -- (2.0,.25) ; % dx
   \node [black, below] at (1.75,0.28) {\fsz $dx$};

   \draw [black] (2.0,0.25) -- (2.0,1.0) ; % dy
   \node [black, right] at (2,0.5) {\fsz $dy\approx 2(x-1)dx$};

   \draw [black] (1.5,0.25) -- (2.0,1.0) ;
\end{tikzpicture}

\caption{\small The length of a small curve segment can be found by Pythagoras' equation. The 
approximation becomes better as $dx$ gets smaller.}
\label{fig-delta_l}
\end{center}
\end{figure}


\section{Parametric Curves}

We now introduce \myindx{parametric curves}. Instead of letting $y$ vary as a function $f(x)$ we now 
let both $x$ and $y$ depend on $t$. This can be visualized by imagining an 'o' drawn with a pen. 
As the pen draws a circle (counter clockwise, starting from the top) you can imagine how the $x$ and $y$ coordinates vary:
First $x$ decreases while $y$ barely changes, then both $x$ and $y$ are decreasing. When we reach the bottom of
the 'o' $x$ is increasing while $y$ only moves slightly, etc. As it takes some time to draw the 'o'
each point is drawn at a certain moment of time, or simply that $\vec{r}(t)= \Big( x(t), y(t) \Big)$.
Let's give some examples. 

The straight line $y=ax+b$ is parametrized as $(t, at+b)$

The curve  given by $y=\sin(x)$ is parametrized by $(t, \sin(t))$

And the general case $y=f(x)$ is parametrized as $(t, f(t))$.\\

So far we have just covered the functions mentioned earlier, but now we can also let the x-coordinate 
vary. For example $(\sin(t),\cos(t))$ draws a circle with radius 1 and center (0,0). For further examples 
see figures \ref{fig-circle2d} and \ref{fig-intersect}.



\begin{figure}
  \begin{tikzpicture}[scale=2]
    \txfig{0}{0}{3.25}{2.5}
    \draw[domain=-3.141:3.141,smooth,red] plot[parametric,prefix=pgffigs/mjc,id=circle] 
            function{1.5+cos(t),1.25+sin(t)};
    \draw[domain=-3.141:3.141,smooth,green!50!black!90] plot[parametric,prefix=pgffigs/mjc,id=sloejfe] 
            function{1.5+cos(t),1.25+sin(t)*cos(t)};
    \node [right] at (.25,2.7) {\fsz $(1.5+\cos t, 1.25+\sin t)$};
    \draw [red] (0.05,2.7) -- (0.249, 2.7);
    \node [right] at (.25,2.4) {\fsz $(1.5+\cos t, 1.25+\sin t\cos t)$};
    \draw [green!50!black!90] (0.05,2.4) -- (0.249, 2.4);
  \end{tikzpicture}

  \caption{\small Circle with radius 1 and center $(1.5,1.25)$ (red) and a self intersecting curve (green).}
  \label{fig-circle2d}
\end{figure}


\begin{figure}
\begin{tikzpicture}[scale=2]
   \txfig{0}{0}{3.25}{2.25}
   \draw[domain=-3.141:3.141,smooth,red] plot[parametric,prefix=pgffigs/mjc,id=intersect] 
            function{1.5+t*cos(t)/2,0.75+t*sin(t)/2};
   \node [left] at (3.5,2.4) {\fsz $(\frac{3}{2}+\frac{t}{2}\cos t, \frac{3}{4}+\frac{t}{2}\sin t)$};
   \fill (1.5,0.75) circle (1pt); 
   \fill (1.5 + 3.141/2 ,0.75) circle (1pt); 
   \fill (1.5 - 3.141/2 ,0.75) circle (1pt); 
   \node [below] at (1.5,.75) {\fsz $t=0$};
   \node [below] at (1.5+3.141/2,.75) {\fsz $t=\pi$};
   \node [below] at (1.5-3.141/2,.75) {\fsz $t=-\pi$};
\end{tikzpicture}
\caption{\small Self intersecting curve showing the points created at $t=-\pi,0$ and $\pi$ respectively. 
$t \in [-\pi;\pi]$}
\label{fig-intersect}
\end{figure}



The length of curves specified as (x,y)=(f(t),g(t)) is given in a similar way as for the simple 
curves mentioned in section \ref{sec:simplefuncs} as we again use Pythagoras. Using the relation

$$
   \Big(\dd{x}{t}dt\Big)^2 + \Big(\dd{y}{t}dt\Big)^2 = \Big(\dot{f(t)}dt\Big)^2 +  \Big(\dot{g(t)}dt\Big)^2
$$
we get for the length of the curve 
\begin{equation} \label{eq:length}
  \mathcal{L} = \int_a^b\sqrt{(dx)^2 + (dy)^2} = 
  \int_a^b\sqrt{(\dot{f})^2 + (\dot{g})^2}dt
\end{equation}

Finally we should mention the parameter $t$. If no limits are specified then we assume that $t$ can take 
the values from $-\infty < t < \infty$. Sometimes, as in the case of the circle this would create 
endless loops around the same curve. In such cases an interval can be specified as $t \in [0;2\pi]$,
meaning that $t$ runs from $0$ to $2\pi$.

\fixme{ Much more can be said about} parametric curves: The speed of traversal,
slopes, asymptotic behavior, turning points, intersection with axes, ...

\section{Tangents}

We have seen that a function has associated with it a $y$-value for every $x$-value. But curves have other 
properties. For example the slope varies with $x$ and this slope is called the \myindx{tangent} 
of the curve, and we can calculate the slope when we know the function.
For functions of the type $y=f(x)$, the slope is defined naturally as the 
small changes in the $y$-direction caused by a small change in $x$. This is 
called the \myindx{derivative} of the function. There are several ways of denoting the 
derivative of a function $y=f(x)$: 

$$
    slope = \dd{f(x)}{x} = f'(x) = f' = y' = \dd{y}{x}
$$ 
and if the variable is measuring time, $t$ as in $y=g(t)$ the derivatime is
sometimes denoted $\dot{g}(t)$ or $\dot{g}$.

When the function is parametrized $(x,y)=(f(t),g(t))$, we differentiate each
function and obtain a \myindx{tangent vector}.

$$
     \vec{v}_t = \begin{pmatrix} f'(t) \\ g'(t) \end{pmatrix}
$$

\begin{figure}[h]
\begin{center}
\begin{tikzpicture}[scale=2]
    \txfig{0}{-0.5}{3.2}{1.4}
    \draw[very thick,color=blue, domain=0:3.1416,smooth] plot[prefix=pgffigs/mjc,id=tangent] function{sin(3*x)/(x**2 +1)};
    \pgfsetlinewidth{0.75pt}
    \draw[red, ->] (0,0) -- +(0.158,0.474);
    \fill (0,0) circle (1pt);

    \draw[red, ->] (0.5,0.798) -- +(0.453,-0.212);
    \fill (0.5,0.798) circle (1pt);

    \draw[red, ->] (0.785,0.437) -- +(0.249,-0.433);
    \fill (0.785,0.437) circle (1pt);

    \draw[red, ->] (1.571,-0.288) -- +(0.484,0.126);
    \fill (1.571,-0.288) circle (1pt);

    \draw[red, ->] (2.356,0.108) -- +(0.486,0.120);
    \fill (2.356,0.108) circle (1pt);

    \node[right] at (2.0,1.2) {\Large $y=\frac{\sin(3x)}{x^2 +1}$};
\end{tikzpicture}
\caption{\small A curve and its tangent vectors at selected points. }
\end{center}
\end{figure}


\section{Curvature}

\begin{figure}[h]
\begin{center}
\begin{tikzpicture}[scale=2.5]
   \txfig{0}{0}{1.75}{1.25}
   \draw[domain=-0.25:2.5,smooth,blue] plot[parametric,prefix=pgffigs/mjc,id=par_test] function{t,(t-1)**2};
   \node [right,blue] at (0.5,1.75) {\fsz $y=(x-1)^2$};
   \pgfsetlinewidth{0.5pt};
   \draw[] (1, 0.707) circle (0.707);
   \draw[] (2.1, 2.1) circle (2.364);
\end{tikzpicture}
\caption{\small Curvature of the function $y=(x-1)^2$ at $x=0$ and $x=1$ respectively. The radius of curvature is indicated. }
\label{fig-curvature}
\end{center}
\end{figure}

\index{curvature!of functions}
The formula for the curvature of a curve of the form $y=f(x)$ is
\begin{equation}
  \kappa=\frac{1}{R^2} = \frac{|y''|}{(1+y'^2)^{3/2}} \label{eq:curv}
\end{equation}
If the curve is parametrized as $(x,y)=(f(t),g(t))$ then the formula
for the curvature is 

\begin{equation}
  \kappa=\frac{|x'y''-y'x''|}{\;(x'^2+y'^2)^{3/2}} 
\end{equation}



\myhrule
\begin{myex}
In figure \ref{fig-curvature}, $y=(x-1)^2$, $y'=2x-2$ and $y''=2$. Using 
equation \ref{eq:curv} we obtain curvature of $y$ as a function of $x$.

$$
\kappa(x) = \frac{2}{(4x^2-8x+5)^{3/2}}
$$

Evaluating this at $x=0$ and $x=1$ we get
$$
  \kappa(0)=\frac{2}{5^{3/2}}\approx 0.179 , \qquad \kappa(1)=2
$$
Using the relation $\kappa=1/R^2$ we can calculate the radius of curvature
at $x=0$ and $x=1$
$$
  R(0)\approx 2.364 , \qquad R(1)\approx 0.707
$$
These are the values used for the two circles in the figure.
\end{myex}

\section{Curves in higher dimensions}

Some of the previous discussions of parametric functions can be 
generalized to higher dimensions. For example the trajectory of 
an object as it travels through 3-dimensional space can be
represented as the position of its center of mass as a function 
of time

$$
   (x,y,z) = \big(X(t), Y(t), Z(t)\big)
$$
whose tangent vector is 
$$
    \vec{v}_t = \big(X'(t), Y'(t), Z'(t)\big)
$$
or if several physical values are measured for a process as a function of time, these 
can be viewed as a parametric curve in n-dimenstions

$$
   \vec{p}(t) = \big(x_1(t), x_2(t), \ldots, x_n(t)\big)
$$
where we have replaced the $x,y,z$ coordinates by $x_1,\ldots,x_n$.

\index{length of!parametric curve}
\index{length of!n-dimensional curve}
A straight forward extension of equation \ref{eq:length} for the length of a parametric curve in 
2 dimensions gives, the length of an n-dimensional curve  

$$
  \mathcal{L} =  \int_a^b\sqrt{\dot{x}_1^2 + \dot{x}_2^2 + \ldots + \dot{x}_n^2 } dt
$$



