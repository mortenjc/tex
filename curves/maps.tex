\chapter{Maps}
\label{maps}

As a short digression we will here describe how coordinate transformations 
have some practical applications within \myindx{navigation}. A position on the surface
of the earth is given by a pair of coordinates called the \myindx{latitude} and \myindx{longitude}.
The process of creating a correspondance between points on the (spherical) earth and 
on a piece of paper (a map) is called a \myindx{projection}.

\section{Mercator projection}

Given the latitude and longitude as $\phi$ and $\lambda$ on the sphere
the \myindx{Mercator projection} provides the following formula for 
the projection

$$
      \vec{r'} = (x',y') = \Big(\lambda-\lambda_0,\; \ln\big(\tan(\frac{\pi}{4}  + \frac{\phi}{2})\big)\Big)
$$


\vspace{0.5cm}
\begin{figure}[!ht]
\twocol{
  \begin{center}
    \begin{pspicture}(-2,-2)(2,2) 
    \psset{unit=2cm}
      \pstThreeDCoor[xMin=-1.5,xMax=1.5, yMax=1.5, zMax=1.5] 
      \pstThreeDSphere[SegmentColor={[cmyk]{0,0,0,0}}](0,0,0){1} 
      \parametricplotThreeD[linecolor=red, algebraic,plotstyle=curve,linewidth=1.5pt](-0.7,2.10)
          {cos(t)| sin(t)|0}
      \parametricplotThreeD[linecolor=red, algebraic,plotstyle=curve,linewidth=1.5pt](0.0,2.40)
          {sin(1.22)*sin(t)| cos(1.22)*sin(t)|cos(t)}
      \parametricplotThreeD[linecolor=blue, algebraic,plotstyle=curve,linewidth=1.5pt](0.6,1.57)
          {0| sin(t)|cos(t)}
      \parametricplotThreeD[linecolor=blue, algebraic,plotstyle=curve,linewidth=1.5pt](0.6,1.57)
          {sin(0.6)*sin(t) | cos(0.6)*sin(t)|cos(t)}
      \parametricplotThreeD[linecolor=blue,algebraic,plotstyle=curve,linewidth=1.5pt](1.0,1.57)
          {cos(t)| sin(t)|0}
      \parametricplotThreeD[linecolor=blue,algebraic,plotstyle=curve,linewidth=1.5pt](1.0,1.57)
          {sin(0.6)*cos(t)| sin(0.6)*sin(t)|cos(0.6)}
      \pstThreeDDot(0,1,0)                   % bottom right
      \pstThreeDDot(1 Cos ,1 Sin,0)          % bottom left
      \pstThreeDDot(0,0.6 Sin,0.6 Cos)       % top right
      \pstThreeDDot(0.6 Sin 1.0 Cos mul,0.6 Sin 1.0 Sin mul,0.6 Cos) % top left
    \end{pspicture}
  \end{center}
}{
  \begin{center}
  \begin{tikzpicture}[scale=0.8]
    \draw[step=.5cm,mjcgrid] (-3.5-0.4,-2.5-0.4) grid (3.5 + 0.4,2.5 +0.4 );
    \pgfsetlinewidth{1pt};

    \draw [->,red] (-3.5,0) -- (3.5,0); 
    \draw [->,red] (0,-2.5) -- (0,2.5); 
    \draw (-3.5,-2.5) -- (-3.5,2.5); 
    \draw (-3.5,2.5) -- (3.5,2.5); 
    \draw (3.5,2.5) -- (3.5,-2.5); 
    \draw (3.5,-2.5) -- (-3.5,-2.5); 
    \draw [blue] (2,0) -- (2,2.0); 
    \draw [blue] (2,2.0) -- (1,2.0); 
    \draw [blue] (1,2.0) -- (1,0); 
    \draw [blue] (1,0) -- (2,0); 

    \fill (1,0) circle (2pt);
    \fill (1,2) circle (2pt);
    \fill (2,0) circle (2pt);
    \fill (2,2) circle (2pt);

    \node [right]  at (3.5,0) {$x'$};
    \node [above]  at (0,2.5) {$y'$};
    \node [below right]  at (-3.5,0) {$-180^\circ$};
    \node [below left]  at (3.5,0) {$180^\circ$};
    \node [below right]  at (0,2.5) {$90^\circ$};
    \node [above right]  at (0,-2.5) {$-90^\circ$};
  \end{tikzpicture}
  \end{center}
}
  \caption{\small Mercator projection. Note how the upper horizontal line is 
stretched so that it has the same length as the lower. This is why areas are
not preserved in the Mercator projection.}
  \label{fig-mercator}
\end{figure}

\fixme{Add points and definitions to maps}

\myhrule
\begin{myex}
What are the coordinates in the Mercator transform of the point on equator on the 
\myindx{Greenwich meridian} $(0^\circ,0^\circ)$? How about the location of Copenhagen $(12^\circ45',55^\circ40')$?

\vspace{0.5cm}
Assuming that we center the map on the Greenwich meridian we set  $\lambda_0=0$. This gives
$(x',y') = (0, \ln(\tan(\pi/4))) = (0,0)$.

\vspace{0.5cm}
For the location of Copenhagen we first represent the coordinates in decimal degrees. Noting that $60'=1^\circ$, 
gives $(12.75^\circ, 55.66^\circ)$, so the longitude is $12.75^\circ$. For the latitude we get $\ln(\tan(45^\circ+55.66^\circ/2))=
1.17$. This a somewhat arbitrary scale where equator is 0 and $45^\circ$ is at $0.88$. Since the North pole will 
be at infinity in this projection some cutoff latitude must be employed. 

\end{myex}
