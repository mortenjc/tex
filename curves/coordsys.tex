\chapter{Coordinate systems}
\label{sec:coordsys}

\section{Basis vectors}
The principal use of a coordinate system is to provide directions i space. A point is clearly located somewhere, but how to define ``somewhere''?

Assume that somehow we have obtained a number of directions (one for the line, two for the plane, three for space, $\ldots$. In the following we assume the dimensionality 2.
We shall call these directions \myindx{basis vectors} $\vec{e}_i$. Every point in space can be described uniquely by 
two numbers $\lambda_i$ associated with the basis vectors


\begin{figure}
\begin{center}
\begin{tikzpicture}[scale=2]
  \draw[step=.25cm,mjcgrid] (-0.4,-0.24) grid (3.4,3.4);
  \pgfsetlinewidth{1pt}

  \def\a{15}
  \coordinate (o)  at (0,0);
  \coordinate (ox) at (3,0);
  \coordinate (oy) at (0,3);
  \coordinate (p)  at (1.5,2);
  \coordinate (c)  at (canvas polar cs:angle=\a,radius=3cm);
  \coordinate (d)  at (canvas polar cs:angle=90-\a,radius=3cm);

  \draw [->,red] (o) -- (c); 
  \draw [->,red] (o) -- (d); 
  \draw [->]     (o) -- (ox); 
  \draw [->]     (o) -- (oy); 
  \pgfsetlinewidth{2pt}
  \draw [->,red] (o) -- (canvas polar cs:angle=\a, radius=1cm); 
  \draw [->,red] (o) -- (canvas polar cs:angle=90-\a, radius=1cm); 
  \draw [->]     (o) -- (1,0); 
  \draw [->]     (o) -- (0,1); 
  \pgfsetlinewidth{1pt}
  \node [right] at (1,2.5)  {$\bar{p}= a\vec{e}_x + b\vec{e}_y = c\vec{e}_x' + d\vec{e}_y'$};

  \pgfsetlinewidth{0.75pt}
  \draw [dotted] (p) -- ($(o)!(p)!(ox)$); 
  \draw [dotted] (p) -- ($(o)!(p)!(oy)$); 
  \draw [dotted,red] (p) -- ($(o)!(p)!(c)$);
  \draw [dotted,red] (p) -- ($(o)!(p)!(d)$);
  \pgfsetlinewidth{1pt}
  \node [below]  at ($(o)!(p)!(ox)$)  {$a$};
  \node [left]   at ($(o)!(p)!(oy)$)  {$b$};
  \node [below,red]  at ($(o)!(p)!(c)$)  {$c$};
  \node [left,red]   at ($(o)!(p)!(d)$)  {$d$};
  \node [right]  at (ox) {$\mathbf{x}$};
  \node [below]  at (0.5,0) {$\vec{e}_x$};
  \node [rotate=\a,right,red] at (c) {\textbf{x'}};
  \node [rotate=-\a,above,red] at (d) {\textbf{y'}};
  \node [above,red]  at (0.6,0.125) {$\vec{e}_x'$};
  \node [right,red]  at (0.125,0.5) {$\vec{e}_y'$};
  \node [above]  at (oy) {$\mathbf{y}$};
  \node [left]   at (0,0.5) {$\vec{e}_y$};
  \fill (p) circle (1pt);
  \fill (o) circle (1pt);
\end{tikzpicture}
\end{center}
\caption{\small The same point will have different coordinates when the basis vectors are different.}
\end{figure}

$$
    \vec{r} = \lambda^1\;\vec{e}_1 + \lambda^2\;\vec{e}_2 \equiv \lambda^i\;\vec{e}_i 
$$
where the summation over all values of $i$ is implicitly assumed.

The basis vectors can be calculated as $\vec{e}_i = \dvd{\vec{r}}{u_i}$. 

\section{The metric tensor}
In order to calculate lengths of curves, areas, angles and curvatures, a mathematical tool 
has shown to be handy. It is called the metric tensor. The \myindx{metric tensor} can
be derived as the dot product of all possible combinations of the basis vectors.
This is written as $g_{ij} =\vec{e}_i\cdot\vec{e}_j$.

Don't worry about the word \myindx{tensor} - so far we can just regard it as a matrix, shown here 
for dimension 3.
$$
   g_{ij} = \begin{pmatrix}
              g_{11} & g_{12} & g_{13} \\
              g_{21} & g_{22} & g_{23} \\
              g_{31} & g_{32} & g_{33} 
            \end{pmatrix}
$$



\vspace{0.5cm}
We here give some examples from two dimensional space on how the metric tensor is invovled in 
evaluating different geometrical properties.

\subsubsection{Lengths}

Let's say we have a point $p$ given as a \myindx{linear combination} of the two 
basis vectors $\eet$ and $\eto$. $p=a \eet + b \eto$

The dot product of $p$ with itself is the square of the length: 
$$
|p|^2 = p\cdot p = a^2 \eets + 2ab\;\eetto + b^2 \etos
= a^2 g_{11} + 2ab\; g_{12} + b^2 g_{22}
$$


\subsubsection{Angles}
For two vectors $p,q$ defined as 
$p=a \eet + b \eto$, $q=c \eet + d \eto$

The dot product of $p$ and $q$ is related to the angle $\alpha$ between them as
$$
  p \cdot q = |p| |q| \cos(\alpha) 
$$

working out the dot product from the definitions of $p$ and $q$ gives
$$
|p| |q| \cos(\alpha) = ac\;\eets + (ad + bc)\eetto + bd\;\etos = A\;g_{11} + B\;g_{12} + C\;g_{22}
$$

\subsubsection{Areas}
For two vectors $p,q$ defined as 
$p=a \eet + b \eto$, $q=c \eet + d \eto$

The length of the cross product is equal to the area spanned by the two vectors, or 
$$
   A^2 = p\times q \cdot p\times q = (a \eet + b \eto)\times(c \eet + d \eto) \cdot (a \eet + b \eto)\times(c \eet + d \eto)
$$
this can be simplified, as $\vec{a}\times\vec{a}=0$, and $\vec{a}\times\vec{b} = -\vec{b}\times\vec{a}$, to
$$
  A^2 = (ad - bc)(\vec{e}_1\times\vec{e}_2)\cdot(\vec{e}_1\times\vec{e}_2) 
$$

$$
  = (ad-bc)[(\eet\cdot\eet)(\eto\cdot\eto) - \eet\cdot\eto] = (ad-bd)[g_{11}^2g_{22}^2 - g_{12}]
$$

\subsubsection{Curvature}

The \myindx{curvature} of a surface with a metric tensor $g_{ij}$ is given as 

$$
  K =  
   \frac{1}{\sqrt{g}} \left[
        \frac{\partial}{\partial u_2} \left( \frac{\sqrt{g}}{g_{11}}\Gamma^2_{11}   \right)
      - \frac{\partial}{\partial u_1} \left( \frac{\sqrt{g}}{g_{11}}\Gamma^2_{12}   \right)
     \right]
$$

where 

$$
    \Gamma^2_{1j} = \frac{1}{2} g^{k2} \left( \frac{\partial g_{1k}}{\partial u^j}  
                 +\frac{\partial g_{jk}}{\partial u^1}
                 -\frac{\partial g_{1j}}{\partial u^k}
       \right)
$$
and
$$
    g = g_{11}\;g_{22} - g_{12}^2
$$


This formula was discovered by Carl Friedrich Gauss, and also has an equivalent in higher 
dimensions. Even if we don't want to calculate the curvature we can 
sometimes recognize a flat surface: If the components of the metric tensor are 
all constant, all derivatives are zero and therefore the curvature is zero and the surface flat.

\subsubsection{Summary}
To summarize: For the purpose of calculating all sorts of geometrical values we need to know the basis vectors, and from them
the metric tensor. These considerations are also valid for higher dimensions.


\section{Cartesian coordinates}
The \myindx{cartesian coordinates}, named after \myindx{Rene Descartes}, are our well known every day
coordinates $x,y$ and $z$.
$$
  \vec{r} = (x,y,z) 
$$

The coordinates can have any value.

$$
   -\infty<x<\infty,\quad  -\infty<y<\infty,\quad  -\infty<z<\infty
$$

The basis vectors are calculated as  

\begin{figure}[h!]
\begin{minipage}{0.5\linewidth}
$$ 
   \eet = \dvd{r}{u_1} =\dvd{r}{x} = \var{1}{0}{0} 
$$ 
$$ 
   \eto = \dvd{r}{u_2} =\dvd{r}{y} =  \var{0}{1}{0} 
$$

$$ 
   \etr = \dvd{r}{u_3} =\dvd{r}{z} = \var{0}{0}{1} 
$$
\end{minipage}
\begin{minipage}{0.5\linewidth}
  \begin{center}
    \begin{tikzpicture} 
      \coordinate (o)  at (0,0,0);
      \coordinate (p)  at (2,2,1);
      \coordinate (x)  at (1,0,0);
      \coordinate (y)  at (0,1,0);
      \coordinate (z)  at (0,0,1);
      \draw [->] (0,0) -- (0,0,2) node [below left] {$x$};
      \draw [->] (0,0) -- (3,0,0) node [right] {$y$}; 
      \draw [->] (0,0) -- (0,2,0) node [above] {$z$}; 
      \draw [red] (o) -- (p);
      \draw [dotted,red] (p) -- (2,0,1);
      \draw [dotted,red] (2,0,1) -- (0,0,1);
      \draw [dotted,red] (2,0,1) -- (2,0,0);
      \fill (p) circle (1pt);

      \pgfusepath{fill}
    \end{tikzpicture}
  \end{center}
  \caption{\small The Cartesian coordinate system. Coordinates are given by $x,y,z$.}
\end{minipage}
\end{figure}




and the metric tensor is
$$
   g_{ij} = \vec{e}_i\cdot\vec{e}_j=  \left( \begin{array}{ccc}
                    1 & 0 & 0 \\
                    0 & 1 & 0 \\
                    0 & 0 & 1 
                    \end{array} 
             \right)
$$
Note that the metric tensor is constant and that all elements outside the diagonal are zero.
This means that the distance between two points differing by $(dx, dy, dz)$ is

$$
   \Delta s = \sqrt{g_{11}dx^2 + g_{22}dy^2 + g_{33}dz^2} = \sqrt{dx^2 + dy^2 + dz^2}
$$


\section{Cylindrical coordinates}
\label{sec:cylcoord}

\myindx{Cylindrical coordinates} are defined as
$$
   \vec{r}(\rho,\theta,z) = (\rho\st, \rho\ct, z)
$$
With the following ranges for the coordinates
$$\rho \ge  0,\quad 0\le\theta\le2\pi, \quad -\infty < z < \infty$$

$$ \eet = \dvd{r}{u_1}=\dvd{r}{\rho}=  \var{\st}{\ct}{0} $$

$$ \eto = \dvd{r}{u_2} =\dvd{r}{\theta} =  \var{\rho\ct}{-\rho\st}{0} $$

$$ \etr = \dvd{r}{u_3}=\dvd{r}{z} =  \var{0}{0}{1}
$$

$$
   g_{ij} = \vec{e}_i\cdot\vec{e}_j =  \left( \begin{array}{ccc}
                    1  & 0      & 0 \\
                    0  & \rho^2 & 0 \\
                    0  & 0      & 1 
                    \end{array} 
             \right)
$$

and that 
$$
   \Delta s^2 = g_{11}d\rho^2 + g_{22}d\theta^2 + g_{33}dz^2 = d\rho^2 + \rho^2d\theta^2 + dz^2
$$


\section{\myindx{Spherical coordinates}}
\label{sec:sphcoord}

$$
  \vec{r} = (x,y,z) = (r \stcp, r \stsp , r\cos\theta)
$$

Where the conventional ranges for the coordinates are

$$
   r\ge0, \quad 0\le\theta\le\pi, \quad 0\le\phi\le 2\pi
$$

The basis vectors are 

\begin{minipage}{0.5\linewidth}
$$
    \eet =  \dvd{r}{r} = \var{\stcp}{\stsp}{\ct}
$$
$$
     \eto =  \dvd{r}{\theta} = \var{r\ctcp}{r\ctsp}{-r\st}
$$
$$
     \etr =  \dvd{r}{\phi} = \var{-r\stsp}{-r\stcp}{0}
$$
\end{minipage}
\begin{minipage}{0.5\linewidth}
    \begin{pspicture}(-2,2)(2,2) 
      \pstThreeDCoor[xMax=2, yMax=2, zMax=2]
      \pstThreeDDot[drawCoor=true](-1,1,1)
     \pstThreeDLine(0,0,0)(-0.4,0.3,0.35)
    \end{pspicture}
\end{minipage}

For this coordinate system the metric tensor can be calculated to 
$$
   g_{ij} = \vec{e}_i\cdot\vec{e}_j  = \left( \begin{array}{ccc}
                    1 & 0 & 0 \\
                    0 & r^2 & 0 \\
                    0 & 0 & r^2\sin^2\theta
                    \end{array} 
             \right)
$$


$$
   \Delta s^2 = g_{11}dr^2 + g_{22}d\theta^2 + g_{33}d\phi^2 = dr^2 + r^2d\theta^2 + r^2\sin^2\theta d\phi^2
$$



\section{Coordinate transformations}

\newcommand{\pyt}{ \sqrt{x^2 + y^2 + z^2} }
Since a point in space can be represented using any regular coordinate system we might find it useful to 
change coordinate system, depending on the purpose. As a natural consequence the need for transforming between coordinate systems arise. 

There exist general methods to transform back and forth between different coordinate systems but
before revealing these a few select cases will be shown.



\subsection{Translated Cartesian}

The simpelst example of a coordinate system transformation is the translated cartesian coordinate system. The
relations between the two coordinate systems are simply

$$
u_i =(x,y,z) = (x' +a, y' + b, z' + c) 
$$
and
$$ 
    u_i'=(x',y',z') = (x -a, y-b, z-c)
$$

\begin{figure}
\begin{center}
\begin{tikzpicture}[scale=2]
  \draw[step=.25cm,mjcgrid] (-0.4,-0.24) grid (3.4,3.4);
  \pgfsetlinewidth{1pt}

  \coordinate (o)  at (0,0);
  \coordinate (ox) at (2,0);
  \coordinate (oy) at (0,2);
  \coordinate (p)  at (1.5,2);

  \draw [->] (o) -- (ox); 
  \draw [->] (o) -- (oy); 
  \draw [->,red] (1,1) -- (1,3); 
  \draw [->,red] (1,1) -- (3,1); 
  \pgfsetlinewidth{2pt}
  \draw [->] (o) -- (1,0); 
  \draw [->] (o) -- (0,1); 
  \draw [->,red] (1,1) -- (1,2); 
  \draw [->,red] (1,1) -- (2,1); 
  \pgfsetlinewidth{1pt}
  \fill (p) circle (1pt);
  \fill (o) circle (1pt);

  \node [right]  at (ox) {$\mathbf{x}$};
  \node [below]  at (0.5,0) {$\vec{e}_x$};
  \node [left]   at (0,0.5) {$\vec{e}_y$};
  \node [above]  at (oy) {$\mathbf{y}$};
  \node [left,red]  at (1,1.5) {$\vec{e}_x'$};
  \node [below,red]  at (1.5,1) {$\vec{e}_y'$};

  \node [right] at (1.5,2.125) {$(1.5,2)$};
  \node [right,red] at (1.5,1.875) {$(0.5,1)$};
 \end{tikzpicture}
\end{center}
\caption{\small Translated (red) cartesian coordinate system}
\end{figure}

\subsection{Rotated Cartesian to Cartesian}
Now lets look at transformation between two cartesian coordinate systems. One is rotated $\alpha$ degrees relative to the other.
$$
    u_i = (x, y, z) = (r\ct, r\st, z)
$$

$$
    u'_i = (x', y', z') = (r'\ct', r'\st', z')
$$

Where $r'=r, z'=z, \theta'=\theta-\alpha$

\begin{center}
\begin{tikzpicture}[scale=2]
  \draw[step=.25cm,mjcgrid] (-1.4,-0.24) grid (3.4,3.4);
  \pgfsetlinewidth{1pt}

  \def\a{20}
  \coordinate (o)  at (0,0);
  \coordinate (ox) at (3,0);
  \coordinate (oy) at (0,3);
  \coordinate (p)  at (1,2);
  \coordinate (c)  at (canvas polar cs:angle=\a,radius=3cm);
  \coordinate (b)  at (canvas polar cs:angle=\a+90,radius=3cm);

  \draw [->] (o) -- (ox); 
  \draw [->] (o) -- (oy); 
  \fill (p) circle (1pt);
  \fill (o) circle (1pt);
  \node [above right] at (p)  {$(x,y)=(1,2)$};

  \draw [<->](2.25,0) arc (0:\a:2.25); 
  \node [left] at(canvas polar cs:angle=\a/2,radius=2.25cm)  {$\alpha$};

  \draw [->,red] (o) -- (c);
  \draw [->,red] (o) -- (b);
  \pgfsetlinewidth{0.75pt}
  \draw [dotted,red] (p) -- ($(o)!(p)!(c)$);
  \draw [dotted,red] (p) -- ($(o)!(p)!(b)$);
  \draw [dotted] (p) -- (1,0.0); 
  \draw [dotted] (p) -- (0,2); 
  \pgfsetlinewidth{1pt}

  \node [rotate=\a,left] at ($(o)!(p)!(b)$) {$y'$};
  \node [rotate=\a,below] at ($(o)!(p)!(c)$) {$x'$};
  \node [rotate=\a,right] at (c) {\textbf{x'}};
  \node [rotate=\a,above] at (b) {\textbf{y'}};

  \node [below]  at ($(o)!(p)!(ox)$)  {$x$};
  \node [left]   at ($(o)!(p)!(oy)$)  {$y$};
  \node [right]  at (ox) {$\mathbf{x}$};
  \node [above]  at (oy) {$\mathbf{y}$};
\end{tikzpicture}
\end{center}

Making use of the trigonometric identities

$$
   \sin(a+b) = \sin(a)\cos(b) + \cos(a)\sin(b)
$$

$$
   \cos(a+b) = \cos(a)\cos(b) - \sin(a)\sin(b)
$$

We get the following transformations between the coordinate systems.

$$
   \var{x'}{y'}{z'} = \left( \begin{array}{ccc}
          \cos\alpha & \sin\alpha & 0\\ 
          -\sin\alpha & \cos\alpha & 0\\
          0 & 0 & 1
       \end{array}  \right)
         \var{x}{y}{z} 
$$
and
$$
   \var{x}{y}{z} = \left( \begin{array}{ccc}
          \cos\alpha & -\sin\alpha & 0\\ 
          \sin\alpha & \cos\alpha & 0\\
          0 & 0 & 1
       \end{array}  \right)
         \var{x'}{y'}{z'} 
$$

This seems right: When $\alpha$ is $0$ then $(x,y,z)=(x',y',z')$. 



\subsection{Cartesian to Spherical}
For example, the connection between Cartesian and Spherical coordinates is given as
$$
    u_i = \begin{pmatrix} x \\  y \\ z \end{pmatrix} = \begin{pmatrix} r\stcp \\ r\stsp \\ r\ct \end{pmatrix}
$$

and

$$
    u'_i = \begin{pmatrix} r \\ \theta \\ \phi \end{pmatrix} 
         = \begin{pmatrix} \pyt \\ \arccos\left(\frac{z}{\pyt}\right) \\ \arcsin\left(\frac{y}{\sqrt{x^2+y^2}}\right) \end{pmatrix}
$$
The value for $\phi$ is only valid when $x \ge 0$, if not then the above value should be subtracted from $\pi$.

\subsection{General Transformations}

\fixme{I'm too tired now...}



\myhrule
\begin{myex}
In the normal cartesian coordinate system $(x,y,z) = (5, 3, \pi)$, what are the coordinates in 
a coordinate system rotated $30^\circ$ counter clockwise?

$(x',y',z') = (5\cos(30)+3\sin(30), 3\cos(30) - 5\sin(30), \pi) \approx (5.83,0.098,\pi)$
\end{myex}
