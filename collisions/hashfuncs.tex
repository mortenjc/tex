\section{On HASH functions}
\IEEEPARstart{H}{ash}
functions are not only used for forwarding descisions, but for packet sampling \cite{Molina}, cryptography 
\cite{Sklavos2003}\cite{Kang2002} and traffic  flow analysis/prioritization \cite{Intel2009}.

 In the previous section we have assumed that 1) the distribution of 
MAC addresses is uniform or 2) that the output of the hashing algorithm
is sufficiently unpredictable. A fundamental paradigm of cryptography is that we can convert any distribution 
into a uniform distribution by means of a sufficiently effective hashing algorithm.
The various cryptographic hashfunctions MD4, MD5, SHA-1, SHA-256, etc. will certainly do the trick,
but would in this case be an overengineered solution.


but these are often too computationally intensive or introduces latency which prevents wire-speed
forwarding.
One report \cite{Stancu2003} shows that if the hashing is poorly done a MAC address size of 4096
entries can be reduced to only 80 entries effectively wasting 98\% of the address table!
In some cases, for example in the case [5], the network designer had full control over the 
distribution of MAC addresses in their network and could alleviate the problem, but this is not 
the case in general.

\subsection{XOR hashing}
The simplest hashing possible is the XOR method mentioned above. Although this method 
is far from random it is often used in practice. The main reason is
its simplicity which lends itself to parallelism, which leads to low latency implementations in ASICS.
XOR based hashing algorithms have been analyzed in \cite{Vandiren2005} and \cite{Martinez2005}.

An interesting side effect of this implementation is the good performance results it gives on standardized
test suites. In particular a switch test suite named UNH
for the institution that tests for interoperability, University of New Hampshire, mandates 
that certain tests uses linearly incrementing MAC addresses. In this case this will lead 
to an essentially perfect utilization of the MAC address table, as will be shows in the simulations.

Computing an 8 bit hash value using this method, one would simply xor together the six bytes of
the MAC address: 
$$h = MAC_1 \oplus MAC_2 \oplus MAC_3 \oplus MAC_4 \oplus MAC_5 \oplus MAC_6$$

instead of performing the xor on the 8 bit bytes, other bit-widths could be used. With some padding scheme 
essentially  any n-bit hash value up to the length (in bits) of the MAC address (and associated data)  can be 
achieved using this method.


\subsection{Pearson hashing}
Pearson hashing is a CBC-MAC using a lookup table $T$, and produces an 8 bit hash
value. Pearson hashing, or some modification of it, is used in at least one series of 
commercial switch products \cite{Intel2009}.
\lstset{language=C}
\begin{lstlisting}
  hash := 0
  for each c in C loop
    index := hash xor c
    hash := T[index]
  end loop
  return hash
\end{lstlisting}
Pearson hashing ensures that not only the individual bytes of the data stream contributes to 
the final hash value. Also their order affects the hash value. This is a necessary property 
in cryptographic systems, but not strictly needed here. Because of this property this 
algorhithm cannot be parallelized.

\subsection{Radix hashing}
Another possibility is to
consider the MAC address array as a six digit Base-R number modulus M.
If the values for R and M are chosen in a suitable manner this method 
should provide a reasonable uniform filling of the bins \cite{Sedgewick2011}. 


\lstset{language=C}
\begin{lstlisting}
  hash := 0
  R := 31
  M := 257
  for each c in C loop
    hash := (R * hash + c) mod M
  end loop
  return hash
\end{lstlisting}

There are a couple of problems with this algorithm. First the value of $M$ should be a prime number in 
order to spread the hash values  evenly, and typically a value which is a power of 2 is desired. Secondly
the method uses the modulus operation which is fairly slow.


